\chapter{Related Work}
\label{ch:related_work} 

\section{Social Media}
Nowadays, there is a large amount of data has been produced by online social media which covered diverse areas such as medicine, history, arts, and it leads to the creation of knowledge by analyzing and clustering those data \cite{jahanbin2020using}. As illustrated before, in this project, our work is to do text mining on social media and discover the potential topics about COVID-19 to analyze the public response during the pandemic. Before this project, the idea of analyzing events by Twitter data cluster has been attempted in many fields such as social interactions, sentiment analysis, and link prediction since there are over 300 million active users and it is open for researchers to access the information on Twitter \cite{martinvcic2017link}. 

As a popular social media outlet, Twitter has become a huge source of linguistic data consist of discussion, opinion and sentiment, and it is considered to have the potential to exert social influence \cite{clark2016sifting}. During times of emergency, Twitter users send tweets with some detailed information which can help the government or researchers to analyze the events and make key decisions\cite{corvey2010twitter}. In \cite{lopez2020understanding}, it proposed a system to discover the public responses to the pandemic and the evolution of the responses in different time by using NLP and text mining through the tweets corpus that related to the COVID-19 pandemic. Based on research about Twitter mention of 6162 COVID-19-related scientific publications \cite{fang2020tracking}, it has been found that the Twitter platform and the users are significant for spreading the research outputs on OVID-19, and the amounts of users that mention COVID-19 are on increase. Therefore, it is visible that social media plays an important role during the COVID-19 pandemic, and Twitter can be chosen as a suitable platform to set up the analysis of the epidemic.

% \section{Text Clustering}


\section{Topic Model}\label{topicmodel}
Topic model is a kind of statistical model to discover the abstract topics in a group of documents, which is commonly used as a text mining tool to extract the latent semantic structures in a text \cite{enwiki:1014350876}. Latent Dirichlet Allocation(LDA) is a classical topic model which has been described as a probabilistic model to process the collections of discrete text data \cite{blei2003latent}. LDA is a three-level hierarchical Bayesian model, and its basic idea is that the topics are the probabilistic distribution over words and the documents are generated by random potential topics. In the existing research \cite{wallach2009rethinking}, LDA has been proved as a useful tool to analyze collections of documents, especially on long text.

According to the statistical analysis, most of the texts from social media contain less than 140 characters \cite{clark2016sifting}. With the plenty of use of the social network in people's daily life, it seems more important for topic models to semantic modeling the short texts on social media. However, short texts generally contain less effective information which makes the features of the sample sparsity, and with the high dimensionality of the feature set, it is hard to extract correct and key features to cluster \cite{tommasel2018short}. Therefore, using traditional topic models such as LDA modeling on the short texts 
to inference the topic distribution, many values of the distribution would be zero, which will lead to the sparsity problem \cite{tang2014understanding}.

Aggregating the short texts to long pseudo-documents is a simple solution to reduce the sparsity problem and has been proved to work better than original LDA \cite{hong2010empirical}. Nevertheless, the effect of the heuristic way to a large extent depends on the quality of data \cite{cheng2014btm}. Adding strong assumptions on the short texts such as assuming each word in the same sentence has the same topic \cite{gruber2007hidden} is also helpful to alleviate the sparsity problem since it can simplify the model. But it might cause the loss of the possibility of capturing multiple topics in a document and might cause overfitting of the model \cite{blei2003latent}. Under this circumstance, Biterm Topic Mode (BTM) is proposed to extract the latent topics in a corpus by modeling the co-occurrence patterns (biterms) in the corpus rather than modeling the single word \cite{cheng2014btm}.

To get a more satisfied result from topic models, some technological improvements have been put on BTM. Li et al. \cite{li2016micro} combined the K-means clustering algorithm with BTM, which uses BTM to attain the topics and uses K-means to cluster those topics better. A novel model named Relation BTM (R-BTM) using word embeddings to link short texts with similar words and enhance the variety of the biterm list \cite{li2019relational}. Moreover, He et al. \cite{he2017fastbtm} proffered a fast BTM to accelerate the sampling process to suit large datasets. As for our model, we extend the biterms (word pairs) to triterms (three-word groups), and modeling the triterms in the corpus, wanted to extract topics efficiently.

\section{Applying to COVID-19}
According to \cite{zheng2020predicting}, there is a hybrid AI model to implement COVID-19 prediction. Firstly, to analyze the variation of the infection rates and detect the spread and the development trend of the disease, an improved susceptible-infected (ISI) model was proffered. This model has been used for predicting some other diseases like SARS and Ebola and has shown strong capabilities in this field. However, those traditional models predict disease by analyzing the dynamic change of the number of diseased individuals and assume that each individual has the same infection rate, which has many limitations and only can generate a general prediction result. Additionally, because of the serious situation during the COVID-19 pandemic and based on some prevented measures and policies from the government and the self-consciousness of citizens, the traditional model cannot meet the prediction requirement of COVID-19. Some news information features and social media information should be considered in this case, so the Natural Language Processing (NLP) is proposed in this epidemic model to obtain a higher accuracy prediction \cite{zheng2020predicting}. 

With the proposal of putting NLP into epidemic models, our project would concentrate on the NLP part of the social media information and not focus on the infection. According to \cite{lee2013real}, there exist a software system to detect the outbreak of the disease based on machine learning and data processing from social media data, whose general framework and goal suit our project. Based on the research \cite{ramirez2017survey, uysal2014impact, jianqiang2017comparison}, the data preprocessing framework has been proposed, which is also helpful to complete this project. 

Thus, our project will learn from the above frameworks of the existing system and different topic models to implement the aims and objectives. The design and implementation details will be discussed in chapter 4\&5. Moreover, Song et al. \cite{song2021classification} applied a topic model on tracking and detecting the disinformation that occurred in the COVID-19 pandemic. And many other systems combined the topic model with other text information such as post time and location to implement a dynamic system to predict the broke out of the epidemic \cite{li2020global}. The future work of our project can be learned from those existing system, so that the topic model can be further and efficiently applied to the COVID-19 epidemic.
